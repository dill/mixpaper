\documentclass[10pt]{article}

% amsmath package, useful for mathematical formulas
\usepackage{amsmath}
% amssymb package, useful for mathematical symbols
\usepackage{amssymb}

% graphicx package, useful for including eps and pdf graphics
% include graphics with the command \includegraphics
\usepackage{graphicx}

% cite package, to clean up citations in the main text. Do not remove.
\usepackage{cite}

\usepackage{color} 

% Use doublespacing - comment out for single spacing
%\usepackage{setspace} 
%\doublespacing


% Text layout
\topmargin 0.0cm
\oddsidemargin 0.5cm
\evensidemargin 0.5cm
\textwidth 16cm 
\textheight 21cm

% Bold the 'Figure #' in the caption and separate it with a period
% Captions will be left justified
\usepackage[labelfont=bf,labelsep=period,justification=raggedright]{caption}

% Use the PLoS provided bibtex style
\bibliographystyle{plos2009}

% Remove brackets from numbering in List of References
\makeatletter
\renewcommand{\@biblabel}[1]{\quad#1.}
\makeatother


% Leave date blank
\date{}

\pagestyle{myheadings}
%% ** EDIT HERE **
%\usepackage{bm}

%% ** EDIT HERE **
%% PLEASE INCLUDE ALL MACROS BELOW

%% END MACROS SECTION

\begin{document}

% Title must be 150 characters or less
\begin{flushleft}
{\Large
\textbf{Text S2: Variance estimation for mixture model detection functions}
}
% Insert Author names, affiliations and corresponding author email.
\\
David L. Miller$^{1,\ast}$,
Len Thomas$^{1}$
\\
\bf{1} School of Mathematics and Statistics, and Centre for Research into Ecological and Environmental Modelling, University of St Andrews, St Andrews KY16 9LZ, Scotland
\\
$\ast$ E-mail: dave@ninepointeightone.net
\end{flushleft}


Variances of both $\hat{N}$ and $\hat{P}_a$ can be estimated for both non-covariate models and covariate models using standard methods \cite{Borchers:2002vc, Marques:2003vb,Borchers:1998uwa}.

In the most general case, the variance of $\hat{N}$ is estimated by:
\begin{equation*}
\hat{\text{Var}}\left( \hat{N} \right) = \left(\frac{\partial \hat{N}}{\partial\hat{\mathbf{\Theta}}}\right)^\text{T}\hat{\mathbf{I}}(\hat{\mathbf{\Theta}})^{-1}\frac{\partial \hat{N}}{\partial\hat{\mathbf{\Theta}}} + \sum_{i=1}^n \frac{(1-p_i)}{p_i^2}
\end{equation*}
where $\hat{\mathbf{I}}(\hat{\mathbf{\Theta}})^{-1}$ is the inverse of the Fisher information matrix,  $\hat{\mathbf{\Theta}}$ is a vector of all of the maximum likelihood estimates of the parameters of the detection function ($\hat{\mathbf{\Theta}}=(\hat{\mathbf{\theta}},\hat{\mathbf{\phi}})$) and all other notation is as in previous sections.

Then for the average detectability:
\begin{align*}
\hat{\text{Var}}\left( \hat{P}_a \right) = \hat{P}_a^2\left\{ \vphantom{\frac{\frac{a}{b}^b}{b}} \frac{\hat{\text{Var}}(\hat{N})}{\hat{N}^2}\right. &+\left.\frac{\left(\frac{\partial \hat{P}_a}{\partial\mathbf{\hat{\Theta}}}\right)^\text{T}\hat{\mathbf{I}}(\hat{\mathbf{\Theta}})^{-1}\frac{\partial \hat{P}_a}{\partial\mathbf{\hat{\Theta}}}+\sum_{i=1}^n (1-p_i)}{n^2} \right. \\
 & \left. - \frac{2\left(\left(\frac{\partial \hat{N}}{\partial\mathbf{\hat{\Theta}}}\right)^\text{T}\hat{\mathbf{I}}(\hat{\mathbf{\Theta}})^{-1}\frac{\partial \hat{P}_a}{\partial\mathbf{\hat{\Theta}}}+\sum_{i=1}^n\frac{(1-p_i)}{p_i^2}\right)}{n\hat{N}}\right\}.\\
\end{align*}


\bibliography{dsmixtures-appendix}

\end{document}