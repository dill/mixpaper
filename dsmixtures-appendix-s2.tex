\documentclass[10pt]{article}

% amsmath package, useful for mathematical formulas
\usepackage{amsmath}
% amssymb package, useful for mathematical symbols
\usepackage{amssymb}

% graphicx package, useful for including eps and pdf graphics
% include graphics with the command \includegraphics
\usepackage{graphicx}

% cite package, to clean up citations in the main text. Do not remove.
\usepackage{cite}

\usepackage{color} 

% Use doublespacing - comment out for single spacing
%\usepackage{setspace} 
%\doublespacing


% Text layout
\topmargin 0.0cm
\oddsidemargin 0.5cm
\evensidemargin 0.5cm
\textwidth 16cm 
\textheight 21cm

% Bold the 'Figure #' in the caption and separate it with a period
% Captions will be left justified
\usepackage[labelfont=bf,labelsep=period,justification=raggedright]{caption}

% Use the PLoS provided bibtex style
\bibliographystyle{plos2009}

% Remove brackets from numbering in List of References
\makeatletter
\renewcommand{\@biblabel}[1]{\quad#1.}
\makeatother


% Leave date blank
\date{}

\pagestyle{myheadings}
%% ** EDIT HERE **
%\usepackage{bm}

%% ** EDIT HERE **
%% PLEASE INCLUDE ALL MACROS BELOW

%% END MACROS SECTION

\begin{document}

% Title must be 150 characters or less
\begin{flushleft}
{\Large
\textbf{Appendix S2: Simulation parameters}
}
% Insert Author names, affiliations and corresponding author email.
\\
David L. Miller$^{1,\ast}$,
Len Thomas$^{1}$
\\
\bf{1} School of Mathematics and Statistics, and Centre for Research into Ecological and Environmental Modelling, University of St Andrews, St Andrews KY16 9LZ, Scotland
\\
$\ast$ E-mail: dave@ninepointeightone.net
\end{flushleft}

In performing the simulations reported in the main article we employed the exponential power series (EPS) and hazard-rate functions as detection function models ("E" scenarios) in order to test the performance of the half-normal mixture models when the candidate model and the model that data was simulated from were not the same. This appendix describes the functional forms of these two functions and lists all parameters values that were used in the simulations (Table 1).

The formulation used for the EPS detection function in simulation E1 was:
\begin{equation*}
g(y,\mathbf{z}; \lambda, b_1) =  \exp(-(y/\lambda)^{-b_1})),
\end{equation*}
which has the following pdf:
\begin{equation*}
f(y,\mathbf{z}; \lambda, b_1) =  \frac{\exp(-(y/\lambda)^{-b_1})),}{\lambda\Gamma(1+\frac{1}{b_1})}
\end{equation*}
where $\lambda=\exp(\beta_1)$ is a scale parameter and $b_1$ is a \textit{shape parameter}.

The formulation for the hazard-rate mixture in simulation E2 was:
\begin{equation*}
g(y,\mathbf{z}; \boldsymbol{\theta}, \boldsymbol{\phi}) = \sum_{j=1}^J \phi_j (1-\exp(-(y/\sigma_j)^{-b_j})),
\end{equation*}
where $b_j$ is the shape parameter associated with the $j^\text{th}$ mixture component.

\begin{table}[htbp]
\centering
\caption{Parameters of the detection functions used in the simulations in Section 3 and the true average detection probability ($P_a$) for each model. For line and point transect models $\beta_1, \beta_2$ and $\pi_1$ are the scale parameters (on the log scale) and mixture proportion. For 3-point simulations, $\beta_3$ is the scale parameter for the third mixture component (again on the log scale) and $\pi_2$ is the second mixture proportion. For covariate models, $\beta_1$ corresponds to the intercept of the first mixture component, $\beta_2$ to the intercept of the second mixture component and $\beta_3$ to the coefficient for the (common) covariate effect. The second (or in the 3-point cases third) mixture proportions can be calculated by summing the values in the table and subtracting the sum from 1. For EPS models, $\beta_1$ is the scale parameter (on the log scale) and $b_1$ is the shape parameter. For hazard-rate models $\beta_1$ and $\beta_2$ are scale parameters (on the log scale) and $b_1$ and $b_2$ is the shape parameters. Numbering is as in Figures 2 and 3 in the main article.}
\begin{tabular}{c c c c c c c c c c}\\
Model & Scenario & $\beta_1$ & $\beta_2$ & $\beta_3$ & $\pi_1$ & $\pi_2$ & $b_1$ & $b_2$ & $P_a$ \\
\hline
Line transect & A1 & -0.223 & -1.897 & &  0.3 & & & & 0.369\\
 & A2 & -0.511 & -2.303 & &  0.7 & & & & 0.514\\
 & A3 &  2.303 & -1.609 & & 0.15 & & & & 0.363\\
 & A4 & -0.357 & -2.996 & &  0.6 & & & & 0.471\\
Point transect &B1 & -0.223 & -1.897 & &  0.3 & & & & 0.24\\
 & B2 & -0.511 & -2.303 & &  0.7 & & & & 0.384\\
 & B3 &  2.303 & -1.609 & & 0.15 & & & & 0.218\\
 & B4 & -0.357 & -2.996 & &  0.6 & & & & 0.378\\
3-point & C1 &  -0.22 &  -0.69 &  -2.3 & 0.3 & 0.3 & & & 0.505\\
 & C2 &   2.71 &  -1.39 &  -3.0 & 0.1 & 0.4 & & & 0.257\\
Covariate & D1 & -2.303 & -0.288 & -0.511 & 0.4 & & & & 0.422\\
 & D2 & -1.609 & -0.223 & -0.916 & 0.4 & & & & 0.389\\
 EPS & E1 & -0.534 & & & & & 1.5& & 0.5 \\
 Hazard-rate & E2 & -1.69 & -0.304 & & 0.5 & & 7 & 7 & 0.5\\
\hline
\end{tabular}
\label{partable}
\bigskip
\end{table}


\bibliography{dsmixtures-appendix}

\end{document}